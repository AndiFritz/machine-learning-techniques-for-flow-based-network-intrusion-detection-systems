\documentclass[notitlepage]{article}

\usepackage{titling}

\usepackage{graphicx}
\usepackage{float}
\usepackage{rotating}
\usepackage[backend=bibtex, sorting=none]{biblatex}
\usepackage{listings}
\usepackage{parskip}

\posttitle{\par\end{center}}
\setlength{\droptitle}{-80pt}

\title{Tussentijdse presentatie}
\author{Axel Faes - 1334986}
\date{Mar 08, 2016}

\begin{document}
\maketitle

aanwezigen: Maarten Wijnants, Peter Quax, Wim Lamotte, Jori Liesenborgs, Wouter vanmontfort, Pieter Robyns, Robin Marx, Bram Bonne, Axel Faes\\

Er is een tussentijdse presentatie geweest waarbij ik mijn huidige progressie moest tonen en een planning moest geven. De presentatie zelf is goed verlopen. Na de presentatie heb ik verschillende vragen gekregen. \\

Veel vragen die gesteld waren, waren bedoeld om te kijken of we het nut/doel van de bachelorthesis kennen en hoe we de invulling correct doen. Ook een belangrijk aspect is hoe het valideren van de correctheid van de experimenten die gedaan zijn/worden zal gebeuren. \\

Een opmerking was dat ik ook bestaande Intrusion detection systemen moet bekijken en ofdat deze machine learning gebruiken. Dan is ook belangrijk waarom ze het wel of niet gebruiken. \\

Er was verwacht dat ik al iets verder stond met de Machine learning cursus. Hierdoor kon ik niet altijd op de volledige diepgang de gestelde vragen beantwoorden. Ik begreep ook niet altijd de onderliggende vraag waardoor ik te oppervlakkig antwoorde. Qua machine learning algoritmes moest ik goed opletten voor overfitting en uitleggen hoe ik hiermee omga.\\

Er zijn ook vragen gesteld m.b.t mijn geplande extra om het intrusion detection systeem real-time te maken. Normaal wordt een flow pas doorgegeven als deze volledig afgesloten is, een mogelijke piste zou zijn om flows al te bekijken ook al zijn ze nog niet afgesloten. Ook het runnen van een VM met malware erop om zelf data-sets te generen is bevestigd dat een goed idee zou zijn. Als ik hiervoor infrastructuur nodig heb moet ik dit vragen.\\

Ik moet opletten met aanvallen die maar zeer weinig netwerktraffiek genereren. Ook moet ik goed beschrijven welke aanvallen wel of niet gedetecteert kunnen worden en uitleggen waarom. Dit staat momenteel al beschreven in mijn thesistekst. Ik zou ook mogelijkheden kunnen uitleggen die ervoor zouden kunnen zorgen dat ik toch alle (of een groot deel) van de aanvallen zou kunnen detecteren. Dit zou bv kunnen door toch packet-data te gaan bekijken.\\

Een algemene opmerking die gegeven was, was dat de presentatie visueler mocht zijn. Figuren en afbeeldingen zijn aangenamer om te tonen aan een publiek. Bij de postersessie moet er ook goed opgelet worden dat ik van persoon tot persoon bekijk hoe diep ik de materie uit mijn bachelorthesis kan uitleggen. 


\end{document}