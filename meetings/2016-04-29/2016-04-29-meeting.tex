\documentclass[notitlepage]{article}

\usepackage{titling}

\usepackage{graphicx}
\usepackage{float}
\usepackage{rotating}
\usepackage[backend=bibtex, sorting=none]{biblatex}
\usepackage{listings}
\usepackage{parskip}

\posttitle{\par\end{center}}
\setlength{\droptitle}{-80pt}

\title{Meeting}
\author{Axel Faes - 1334986}
\date{April 29, 2016}

\begin{document}
\maketitle

aanwezigen: Bram Bonne, Axel Faes \\
\\
Het herschrijven van het "Algorithms" hoofdstuk heeft langer geduurd dan gedacht en is pas ingeleverd op woensdag 27 april 2016. Tijdens de meeting is toegelicht welke aanpassingen gebeurt zijn aan de thesis. Er is kort besproken dat er ook focus gelegd moet worden over het verwerken van de data van Cegeka. Dit zou al belangrijk zijn voor de eerste draft van de thesis. \\
\\
De actiepunten die gedaan zijn:
\begin{itemize}  
        \item Beginnen met verwerken van data van Cegeka
        \item Schrijven korte summary op het einde van het machine learning hoofdstuk
        \item Numerieke voorbeelden plaatsen bij regularisatie en feature scaling
        \item Hoofdstuk "Algorithms" herwerken
\end{itemize}

Volgende actiepunten zijn besproken:
\begin{itemize}  		
		\item Schrijven hoofdstuk IP Flows
		\item Herwerken van hoofdstuk "Machine learning for an IDS"
        \item Verder verwerken dataset Cegeka
\end{itemize}

\end{document}