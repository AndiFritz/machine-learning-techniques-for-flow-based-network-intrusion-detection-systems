\documentclass[notitlepage]{article}

\usepackage{titling}

\usepackage{graphicx}
\usepackage{float}
\usepackage{rotating}
\usepackage[backend=bibtex, sorting=none]{biblatex}
\usepackage{listings}
\usepackage{parskip}

\posttitle{\par\end{center}}
\setlength{\droptitle}{-80pt}

\title{Meeting}
\author{Axel Faes - 1334986}
\date{Mar 04, 2016}

\begin{document}
\maketitle

aanwezigen: Bram Bonne, Axel Faes

Deze week is voornamelijk besteed aan de thesis en aan het leren van de machine learning cursus. De machine learning cursus is gevolgd tot hoofstuk 10. De cursus zou tegen volgende meeting af moeten zijn. De algemene structuur van de thesistekst is nagekeken. Het hoofdstuk over "Attack classification" moet uitgebreid worden met een algemene uitleg over hoe aanvallen gedetecteerd kunnen worden. Het hoofdstuk over de gebruikte data-sets moet samengevoegd worden met het hoofdstuk dat de implementatie beschrijft. Het hoofdstuk over voor/nadelen van machine learning voor intrusion detection systemen moet verwerkt worden in het algemene hoofdstuk over machine learning.

De actiepunten die gedaan zijn:
\begin{itemize}  
		\item Verder werken aan ML cursus
        \item Schrijven aan thesis.
\end{itemize}

Volgende actiepunten zijn besproken:
\begin{itemize}  		
\item Verder werken aan ML cursus
        \item Schrijven aan thesis.
\end{itemize}

\end{document}