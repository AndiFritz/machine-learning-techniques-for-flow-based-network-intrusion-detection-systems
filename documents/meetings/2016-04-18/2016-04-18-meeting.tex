\documentclass[notitlepage]{article}

\usepackage{titling}

\usepackage{graphicx}
\usepackage{float}
\usepackage{rotating}
\usepackage[backend=bibtex, sorting=none]{biblatex}
\usepackage{listings}
\usepackage{parskip}

\posttitle{\par\end{center}}
\setlength{\droptitle}{-80pt}

\title{Meeting}
\author{Axel Faes - 1334986}
\date{April 18, 2016}

\begin{document}
\maketitle

aanwezigen: Bram Bonne, Axel Faes \\
\\
Ik heb mijn huidige vooruitgang laten zien van de bachelorproef tekst. Hierbij is een korte herschikking gekomen van de hoofdstukken. Ik moet het "Implementatie" hoofdstuk opsplitsen naar een hoofstuk "Implementatie" en een hoofdstuk "Evaluatie". In "Implementatie" moet mijn implementatie zelf beschreven staan, in "Evaluatie" moeten de resultaten beschreven worden. \\\\
Verder is kort uitgelegd wat de algemene structuur is van het machine learning hoofdstuk en welke aanpassingen er gebeurd zijn. Het hoofdstuk is opgesplitst in meerdere delen zodanig dat de uitleg, de algoritmes en validatie in aparte hoofdstukken uitgelegd wordt. Het hoofdstuk "Preventie" moet ik aanpassen naar iets gelijkaardigs aan "Future work". Het hoofdstuk "Visualisatie" moet samengevoegd worden met "Implementatie".
De actiepunten die gedaan zijn:
\begin{itemize}  
        \item  Verwerken feedback op machine learning hoofdstukken.
\end{itemize}

Volgende actiepunten zijn besproken:
\begin{itemize}  		
		\item Herschrijven en verwerken van feedback op de thesistekst
        \item Halen data bij Cegeka
\end{itemize}

\end{document}