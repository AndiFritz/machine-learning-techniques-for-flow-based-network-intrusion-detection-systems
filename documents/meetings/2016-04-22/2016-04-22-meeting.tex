\documentclass[notitlepage]{article}

\usepackage{titling}

\usepackage{graphicx}
\usepackage{float}
\usepackage{rotating}
\usepackage[backend=bibtex, sorting=none]{biblatex}
\usepackage{listings}
\usepackage{parskip}

\posttitle{\par\end{center}}
\setlength{\droptitle}{-80pt}

\title{Meeting}
\author{Axel Faes - 1334986}
\date{April 22, 2016}

\begin{document}
\maketitle

aanwezigen: Bram Bonne, Axel Faes \\
\\
Ik heb mijn huidige vooruitgang laten zien van de bachelorproef tekst. De meeste tekst is al stukken beter. Er mogen bij enkele stukken nog numerieke voorbeelden komen te staan. Deze stukken zijn de regularisatie en de feature scaling. Het hoofdstuk over "Algorithms" moet nog herschreven worden. Tevens ga ik een "Summary" schrijven op het einde van het machine learning hoofdstuk zodanig dat nog snel en kort een samenvatting gegeven wordt.\\
\\
De actiepunten die gedaan zijn:
\begin{itemize}  
        \item  Verwerken feedback op machine learning hoofdstukken.
\end{itemize}

Volgende actiepunten zijn besproken:
\begin{itemize}  		
		\item Verwerken feedback en herschrijven "Algorithms" tegen zondag 24 April
        \item Schrijven van tekst
\end{itemize}

\end{document}