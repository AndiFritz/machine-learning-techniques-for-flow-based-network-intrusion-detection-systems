\documentclass[notitlepage]{article}

\usepackage{titling}

\usepackage{graphicx}
\usepackage{float}
\usepackage{rotating}
\usepackage[backend=bibtex, sorting=none]{biblatex}
\usepackage{listings}
\usepackage{parskip}

\posttitle{\par\end{center}}
\setlength{\droptitle}{-80pt}

\title{Meeting Cegeka}
\author{Axel Faes - 1334986}
\date{April 06, 2016}

\begin{document}
\maketitle

aanwezigen: Peter Quax, Axel Faes, Cegeka

Er is een meeting geweest met Cegeka om de mogelijkheid te bespreken om data te kunnen gebruiken die afkomstig is van Cegeka, alsook om informatie te krijgen van de hudige IDS' die gebruikt worden. \\\\
In het begin is een korte presentatie gegeven waarin de eigenschappen van het te ontwikkelen systeem uitgelegd worden. Er wordt ook beschreven wat de algemene onderzoeksvragen zijn en hoe data nu gebruikt kan worden in machine learning algoritmes. \\\\
Een eerste vraag die gesteld is, is welke classificatie van ‘onverwachte traffiek’ er momenteel gebeurt in datacenter/hosting context. Cegeka werkt heel gelaagd. Voor de routers staat een DDoS protection systeem. Na de router staan zowel firewalls als SIEM devices. Voor grotere klanten is er extra beveiliging voorzien in de vorm van afgestelde IPS systemen en meer gedetailleerde threat detection. Zoveel mogelijk data wordt verwerkt, zowel flows als meer gedetailleerd. De systemen werken voornamelijk met een signature database en verwerken de data voornamelijk automatisch. De systemen moeten wel manueel afgesteld en onderhouden worden. \\\\
De classificatie gebeurt heel gedetailleerd door deze verschillende lagen. Er werd wel gesteld dat het veel interessanter is om outbound verkeer na te kijken in vergelijking met binnengaand verkeer. Zaken zoals port-scans zijn interessant om te weten maar gebeuren heel veel en kunnen al goed gedetecteerd worden. \\\\
Er was veel interesse naar een intrusion detectie systeem dat op basis van netflow en machine learning technieken werkt. Er accuraatheid van rond de 70-80 procent zou gezien worden als een goede accuraatheid. Ze hebben ook liever false positives dan false negatives. Teveel alerts genereren is heel vervelend, maar het is belangrijker dat voldoende anomalieen gedetecteerd worden. \\\\
Er is gevraagd of het mogelijk is om data te verkrijgen. Dit was zeker mogelijk. De netflow en corresponderende logs van 3 dagen wordt geleverd. De logs worden zowel in binair als text formaat geleverd. Het binair formaat kan ingelezen door een programma dat de logs visualiseerd. Dit programma wordt ook meegeleverd. Mogelijks zou ook output van het DDoS systeem geleverd kunnen worden. Om te bekijken ofdat een flow gezien is als malicious of niet moet dit bekeken worden of deze flow voorkomt in de logs.

\end{document}