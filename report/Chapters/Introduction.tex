% Chapter 1

\chapter{Introduction} % Main chapter title

\label{Chapter1} % Change X to a consecutive number; for referencing this chapter elsewhere, use \ref{ChapterX}

The internet is constantly growing and new network sevices arise constantly. Sensitive data is also increasingly being stored digitally. All these new services could contain security flaws which could leak private data, such as passwords or other sensitive data. This means that security flaws become more and more important since they can cause so much damage. It is not just the leaking of sensitive data that is an issue, but also protecting a computer or network against malware is important. \\
\\
For example, earlier this year, the internal network of a hospital was attacked by a ransomware attack. The attackers did not get access to personal information and the hospital was able to go back to doing paperwork. But it does show that cyber attacks are a serious threat. \cite{ransomware}\\
\\
Considering this, it becomes more important to be able to detect and prevent attacks on network systems. Intrusion detection systems are used for this purpose. An intrusion detection system can alert administrators of malicious behaviour. Intrusion detection systems can detect attacks using several different methods as explained in Section~\ref{ids}. \\
\\
These methods are not always reliable. They may not catch every attack in which case the intrusion detection system is not that usefull. They could also give alerts for events that are not malicious at all. This could desensitize administrators for the alerts an intrusion detection system gives. \\
\\
In order to have good performance, most intrusion detection systems need a lot of manual maintenance. This is needed to constantly check the alerts the intrusion detection system gives and tweak the system if required. Other systems which are already well configured have a big price tag. This thesis tries to find out \textit{whether an intrusion detection system can work out-of-the-box with an acceptable performance.} \\
\\
This is done by using machine learning algorithms. These are algorithms which can learn and find patterns in input. They can find out what different types of behaviour look like. This means that they could for example learn what botnet behaviour looks like. Machine learning algorithms are further explained in Section~\ref{Chapter2}. \textit{Machine learning algorithms seems promising for the problem of automatic intrusion detection, but is this true?} However, there are different kinds of machine learning algorithms. \textit{These different kinds of algorithms will be discussed and evaluated to see whether they are useful in anomaly detection.} \\
\\
The fact that the internet is growing continuously, this means that more and more data passes through the networks of big companies and data centers. This makes it quite difficult to be able to check every piece of data that passes through the network. This could be solved by using IP Flows. These aggregate a lot of packet data into a small flow, they are further explained in Section~\ref{flow}. These flows are small, but also contain less information than the complete packet data. \textit{But is it possible to still detect attacks when only the IP flows are used?} \\
\\
It is not just important to find out whether attacks can still be detected when only IP flows are used. It is also important to know which kind of anomalies can be detected. \textit{Could every type of attack be detected or is it a subset? } \\
\\
These questions lead to the development of a system which uses machine learning techniques in a purely flow based intrusion detection system. Such a system would be one of the first lines of defense of a company against cyber attacks. Real world data is going to be used to test \textit{whether the system can operate in a real life situation.}

