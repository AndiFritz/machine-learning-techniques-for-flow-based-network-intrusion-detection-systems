% Chapter Template

\chapter{IP Flows} % Main chapter title

\label{flow} % Change X to a consecutive number; for referencing this chapter elsewhere, use \ref{ChapterX}

\label{export}
Flows are aggregated from all packet data that travels through the network. Flow exporters are programs which collect network packets and aggregate them into flow records. A flow is not the same as a TCP connection. A flow can be any communication between two devices with any protocol. Flows are defined using a (source\_IP, destination\_IP, protocol) tuple. This is why flows are also called IP Flows.\\
\\
Since flow data does not contain any payload information, intrusion detection systems that use flow data cannot detect malicious behaviour embedded within payload data. \cite{IPFlow}

\section{How to use flow-data}
The following attributes are available with flow-data:
\begin{itemize}
\item Source IP
\item Destination IP
\item Protocol name
\item Source port
\item Destination port
\item Starting time of the flow
\item Duration of the flow
\item Amount of packets in the flow
\item Amount of bytes in the flow
\end{itemize}
However, should a flow exporter be implemented, some additional features can be generated from packet data. \ref{export}
\begin{itemize}
\item Amount of TCP SYN within the flow
\item Source and Destination Type of Service
\item Payload size
\end{itemize}
These data can be used within the machine learning algorithms. However some variables have undesirable effects on the accuracy of the algorithm. Some care should be taken when training the machine learning algorithms with the additional data. Not all data, both training data as predictive data, will have the additional features.\\
\\
Most machine learning libraries use numeral data instead of string data. All string data has been hashed in order to be able to use it in machine learning algorithms. The probability on a collision is low enough to be able to ignored.

\subsection{IP addresses}
Flow data can contain multiple forms of IP addresses. Both IPv4 and IPv6 data can be found. For some protocols the flow data can also contain the MAC-addresses instead of the IP-address. These addresses are hashed, so they become numeral, discrete data and are then fed into the machine learning algorithm.\\
\\
Using the IP, it is possible to find the country or region of origin. Tests where the country of origin was fed into the machine learning algorithms, have been done. Results however showed, that the accuracy of the IDS became lower.

\begin{table}[H]
\caption{The effects of using IP country-of-origin on accuracy of IDS.}
\label{tab:country}
\centering
\begin{tabular}{l l l}
\toprule
\tabhead{With Country-of-origin} & \tabhead{Accuracy}\\
\midrule
Yes & 96.16\%\\
No & 98.57\%\\
\bottomrule\\
\end{tabular}
\end{table}

\subsection{Ports and protocol name}
Both the source and destination port are discrete data. They are usually received in decimal form, however some data-sets might use them in hexadecimal data or refer to ports as "ssh port" instead of "22". Port data, in decimal form, can be directly fed into the machine learning algorithm.\\
\\
The protocol name can simply be converted to a standard string in lower case, in order to avoid errors by lower and uppercase forms of the same name (for example "tcp" and "TCP"). This string can than be hashed into a discrete value.

\subsection{Timing}

\subsection{Size} 
The amount of packets used in the flow and the amount of bytes are both discrete data. They are always received in decimal form. They can immediately be fed into the machine learning algorithm.
