% Chapter 1

\chapter{Attack Classification} % Main chapter title

\label{attack} % Change X to a consecutive number; for referencing this chapter elsewhere, use \ref{ChapterX}

An intrusion detection system can use multiple methods to detect malicious behaviour.  Since flow-based intrusion detection systems only have access to the flows and not the payload, they cannot detect every kind of attack. In order to make the IDS as effective as possible, the exact classifications of attacks that can be detected need to be known.

\section{Classification}
There are several types of attacks that can occur. Some of these attacks occur only on the network, other attacks infect computers, called malware. The exact classifications are not mutually exclusive. Some types of malware utilise network attacks. However it is important to make a distinction between these attacks. Every attack is identified by different characteristics. Knowing these characteristics is usefull to be able to tweak the IDS to make identification more effective.

\section{network attacks}
There are \textbf{Physical attacks}, these are attacks which attempt to destroy physical equipment and hardware. \textbf{Buffer overflows} are attacks that try to execute arbritrairy code or crash a process by overflowing a buffer on the targeted system. \textbf{Password attacks} attempts to break into a system by gaining the password that the system uses. The simplest password attacks are brute-force password crackers. \textbf{DDOS} attacks are attacks which attempt to make a network resource temporarily or permanently unavailable for the users of that resource. An attack could happen by flooding a system with TCP SYN packets. \textbf{Network scans} are information gathering attacks. They do not cause any damage by themselves but usually serve the purpose to gather information about a system that could be used in further attacks. Network traffic sniffing or port scans are examples of network scans.

\section{Malware}
There are several types of malware. We can make four distinct categories of malware. There are \textbf{botnets}, \textbf{viruses}, \textbf{trojan horses} and \textbf{worms}. Malware are actual programs that infect a system to execute a specific task. The task of the malware defines which categorie the malware belongs in.\\
\\
\textbf{Trojan horses} are programs disguised as harmless applications but contain malicious code. \textbf{Worms} are programs that replicate themselves among a network.  They can spread extremely fast. \textbf{Viruses} are similar to worms. However they only replicate themselves on the infected host computer. Thus they require user interaction in order to be spread around a network. The virus can accomplish this by attaching itself to an email-attachment, embed itself within an executable, etc. \\
\\
\textbf{Botnets} is malware that causes infected computers to become "slaves" to the master. An infected computer is controlled externally by the bot-master without the knowledge of the owner of the infected computer. The bot-master can use the distributed network of "slave" computer to perform other malicious tasks, such as performing an DDOS attack.

\section{Detection}
An NIDS only monitors the network. As such not every attack can be detected by an NIDS. Only the attacks that actually use the network can be detected. Flow-based IDS have the additional constraint that they can only use flow data. This further limits the attacks that can be detected. The attacks that can be detected using a flow-based network intrusion detection systems are:
\begin{itemize}
\item DDOS
\item Network scans
\item Worms
\item Botnets
\end{itemize}
Other attacks either do not use network communication, or they are not visible within the header information of network traffic. 

\subsection{Distributed Denial of Service}
A distributed denial of service can be detected by the amount of data that is being received. However, there are many different types of DDoS attacks. There are ICMP floods, SYN floods, etc. These attacks can be described in terms of traffic patterns. A traffic pattern is expressed in a couple features. These features include the number of flows and packets, the packet size, and the total bandwidth used during the traffic. For example UDP flooding can be characterised by a traffic pattern which contains a lot of packets. These patterns can be searched for during the detection phase. \cite{kim2004flow}
\section{Network scans}

\section{Worms}

\section{Botnets}

